\documentclass[12pt,a4paper,oneside]{article}

\usepackage[margin=3cm]{geometry}

\usepackage{hyperref}
\hypersetup{
    pdftitle={COM 254, Mobile and IoT Application Development},%
    pdfauthor={Toksaitov Dmitrii Alexandrovich},%
    pdfsubject={Syllabus},%
    pdfkeywords={COM;}{254;}{syllabus;}{mobile;}{IoT;}{application;}{development},%
    colorlinks,%
    linkcolor=black,%
    citecolor=black,%
    filecolor=black,%
    urlcolor=black
}

\newcommand{\R}[1]{\uppercase\expandafter{\romannumeral #1\relax}}

\begin{document}

    \title{COM 254, Mobile and IoT Application Development}
    \author{
        American University of Central Asia\\
        Software Engineering Department
    }
    \date{}
    \maketitle

    \section{Course Information}

        \begin{description}
            \item[Course ID]\hfill\\
                COM 254, 4389
            \item[Course Repositories]\hfill\\
                \url{https://github.com/auca/com.254}
            \item[Class Discussions]\hfill\\
                \url{https://piazza.com/auca.kg/spring2018/com254}
            \item[Place]\hfill\\
                AUCA, laboratory G31
            \item[Time]\hfill\\
                Tuesday 14:10\\
                Thursday 14:10
        \end{description}

    \section{Prerequisites}

        COM 112, Programming \R{2}

    \section{Contact Information}

        \begin{description}
            \item[Instructor]\hfill\\
                Toksaitov Dmitrii Alexandrovich\\
                \href{mailto:toksaitov_d@auca.kg}{toksaitov\_d@auca.kg}
            \item[Office]\hfill\\
                AUCA, room 315\\
                AUCA, Media Laboratory
            \item[Office Hours]\hfill\\
                Tuesday 15:35--18:00\\
                Thursday 15:35--18:00
        \end{description}

    \section{Course Overview}

This course introduces students to development tools and APIs to build
applications for the Google Android operating system to manage
networks of physical devices, vehicles, home appliances, and other
items embedded with electronics, sensors, and actuators. Students will
get introduced to embedded development on the Arduino platform with
the help of ESP8266, ESP32, and several other WiFi, Bluetooth, and
LoRa-enabled chips with programmable microcontrollers. Students will
also learn how to build unique interactive user interfaces for
multi-touch mobile devices on the Android platform to manage embedded
devices around us. The mobile development part covers object-oriented
design using the Model-View-Controller paradigm, the Java programming
language for the Android Runtime, development frameworks, device
emulators, and application build tools. Other topics include
multi-threading, power and performance considerations, the accelerated
2-D and 3-D graphics APIs. By bringing two platforms together,
students will prototype appliances that can be controlled through
mobile phones to help people with their daily life. The course
projects range from building a simple smart light bulb to an automatic
data collection system with a toy car robot for an indoor positioning
system.

    \section{Topics Covered}

        \begin{itemize}
            \item Development tools (Android Studio, SDK, device emulators)
            \item App. fundamentals (activities, services, content providers)
            \item User interface elements
            \item Graphics and animation
            \item Data storage
            \item Connectivity
            \item Media and camera
            \item Working with device sensors
            \item Publishing and distributing applications
            \item Basics of Digital Electronics
            \item The Arduino IDE
            \item Working with the ESP8266, ESP-32 boards
        \end{itemize}

    \section{Practice Tasks}

        Students are required to finish 10 practice tasks. The tasks are based
        on topics discussed during lectures. Each task should be finished during
        the class to receive a grade.

    \section{Course Projects}

        Each student will have to develop an app for the Android platform and a connected device. The
        challenge of the project is to maintain a certain level of quality for
        the application to be able to publish it to end users on Google Play
        Store at the end of the course.

    \section{Final Exam}

        At the end of the course, students have to take the final examination in
        a form of a quiz with a number of multiple choice questions on topics
        discussed during classes.

    \section{Reading}

        \begin{enumerate}
            \item Introduction to Android Application Development: Android
            Essentials, 5th Edition by Joseph Annuzzi Jr., Lauren Darcey, Shane
            Conder (ISBN: 978-0134389455)
            \item Java: A Beginner's Guide, 6th Edition by Herbert Schildt
            (ISBN: 978-0071809252)
        \end{enumerate}

        \subsection{Supplemental Reading}

            \begin{enumerate}
                \item Design Patterns: Elements of Reusable Object-Oriented
                Software by Erich Gamma, Richard Helm, Ralph Johnson, John
                Vlissides (AUCA Library Call Number: QA 76.64 D47 1995, ISBN:
                978-0201633610)
                \item Refactoring: Improving the Design of Existing Code by
                Martin Fowler, Kent Beck, John Brant, William Opdyke, Don
                Roberts (AUCA Library Call Number: QA76.76.R42 F695 1999, ISBN:
                978-0201485677)
            \end{enumerate}

    \section{Grading}

        \begin{itemize}
            \item Practice tasks (30\%)
            \item Course project (40\%)
            \item Final examination (30\%)
        \end{itemize}

        \begin{itemize} \itemsep-10pt \parskip0pt \parsep0pt
            \item[--] 90\%--100\%: A\\
            \item[--] 80\%--89\%: A-\\
            \item[--] 70\%--79\%: B+\\
            \item[--] 65\%--69\%: B\\
            \item[--] 60\%--64\%: B-\\
            \item[--] 56\%--59\%: C+\\
            \item[--] 53\%--55\%: C\\
            \item[--] 50\%--52\%: C-\\
            \item[--] 46\%--49\%: D+\\
            \item[--] 43\%--45\%: D\\
            \item[--] 40\%--42\%: D-\\
            \item[--] Less than 39\%: F
        \end{itemize}

    \section{Rules}

        Students are required to follow the rules of conduct of the Software
        Engineering Department and American University of Central Asia.

        Team work is NOT encouraged. Equal blocks of code or similar structural
        pieces in separate works will be considered as academic dishonesty and
        all parties will get zero for the task.

\end{document}
